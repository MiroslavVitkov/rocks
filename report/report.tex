\documentclass{article}


\usepackage[table]{xcolor}


\title{Comparisson of methods for multiclass classification of multi-dimensional data}
\author{Miroslav Vitkov}


\begin{document}
\maketitle

\section{Introduction}
The task is to classify rock samples in one of six classes.
The input datapoints are 7810-element arrays.
The SVM model showed best promise.


\section{Dataset}
The dataset consists of 2362 datapoints all classified within the 6 labels and 45 sublabels.
The reference number of the dataset is 190401.


\section{Preprocessing}
The data is not being preprocessed, despite obvious indicators of noise.


\section{Models}
The following algorithms were tried with default parameters from the corresponding libraries implementing them.
Accuracy is calculated as:
Prediction time is:
Training time is:
Computational mode illustrates how much room for improvement there is in the used implementation.
\\ \par
\rowcolors{1}{white}{lightgray}
\begin{tabular}{ c | c | c | c | c }
algorithm      & accuracy & prediction time & training time & computational model \\
random chance  & & & & single thread \\
correlation    & & & & multithreaded \\
SVM            & & & & GPU \\ verify!!
neural network & & & & GPU \\  verify!!
random forest  & & & & multithreaded \\
\end{tabular}


\subsection{Random Chance}
Firstly, for sanity checking of the system, a random chance model was developed.
It looks only at the distribution of training labels and produces a similar stochastic distribution at prediction time.


\section{Correlation}
With such a small dataset, correlation is feasible.
It yields good results, but is not scalable.


\section{SVM}
The SVM model performs well and runs quickly at prediction time.
%maximum-margin classifier
%kernel trick
%hard/soft margin
%Radial basis function kernel
%http://webspace.ulbsibiu.ro/lucian.vintan/html/sci.pdf
%one vs one / one vs many
%curse of dimensionality
%https://stats.stackexchange.com/questions/77876/why-would-scaling-features-decrease-svm-performance?rq=1
%https://arxiv.org/pdf/0810.4752v1.pdf
%https://stats.stackexchange.com/questions/10423/number-of-features-vs-number-of-observations/10426#10426
%https://stats.stackexchange.com/a/10426/208261
%https://stats.stackexchange.com/questions/186184/does-dimensionality-curse-effect-some-models-more-than-others
%choice of regularization coefficient c
%choice of kernel
%choice of kernel parameters
% probabilistic classifier

\section{Neural Network}
A naive model with two hidden layers with relu activations performed extremely poorly: <>.
It is unclear if this is due to programming errors or due to wrong architecture of the network.


\section{Random Forest}
A OpenMP-based random forest classifier took extremely long to train and yielded the following result:


\section{Conclusion}
An SVM is a suitable baseline model for multiclass classification of high dimensionality objects.


\end{document}

